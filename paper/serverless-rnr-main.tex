%% For double-blind review submission, w/o CCS and ACM Reference (max submission space)
\documentclass[sigplan,10pt,review,anonymous]{acmart}\settopmatter{printfolios=true,printccs=false,printacmref=false}
%% For double-blind review submission, w/ CCS and ACM Reference
%\documentclass[sigplan,10pt,review,anonymous]{acmart}\settopmatter{printfolios=true}
%% For single-blind review submission, w/o CCS and ACM Reference (max submission space)
%\documentclass[sigplan,10pt,review]{acmart}\settopmatter{printfolios=true,printccs=false,printacmref=false}
%% For single-blind review submission, w/ CCS and ACM Reference
%\documentclass[sigplan,10pt,review]{acmart}\settopmatter{printfolios=true}
%% For final camera-ready submission, w/ required CCS and ACM Reference
%\documentclass[sigplan,10pt]{acmart}\settopmatter{}


%% Conference information
%% Supplied to authors by publisher for camera-ready submission;
%% use defaults for review submission.
\acmConference[PL'17]{ACM SIGPLAN Conference on Programming Languages}{January 01--03, 2017}{New York, NY, USA}
\acmYear{2017}
\acmISBN{} % \acmISBN{978-x-xxxx-xxxx-x/YY/MM}
\acmDOI{} % \acmDOI{10.1145/nnnnnnn.nnnnnnn}
\startPage{1}

%% Copyright information
%% Supplied to authors (based on authors' rights management selection;
%% see authors.acm.org) by publisher for camera-ready submission;
%% use 'none' for review submission.
\setcopyright{none}
%\setcopyright{acmcopyright}
%\setcopyright{acmlicensed}
%\setcopyright{rightsretained}
%\copyrightyear{2017}           %% If different from \acmYear

%% Bibliography style
\bibliographystyle{ACM-Reference-Format}
%% Citation style
%\citestyle{acmauthoryear}  %% For author/year citations
%\citestyle{acmnumeric}     %% For numeric citations
%\setcitestyle{nosort}      %% With 'acmnumeric', to disable automatic
                            %% sorting of references within a single citation;
                            %% e.g., \cite{Smith99,Carpenter05,Baker12}
                            %% rendered as [14,5,2] rather than [2,5,14].
%\setcitesyle{nocompress}   %% With 'acmnumeric', to disable automatic
                            %% compression of sequential references within a
                            %% single citation;
                            %% e.g., \cite{Baker12,Baker14,Baker16}
                            %% rendered as [2,3,4] rather than [2-4].


%%%%%%%%%%%%%%%%%%%%%%%%%%%%%%%%%%%%%%%%%%%%%%%%%%%%%%%%%%%%%%%%%%%%%%
%% Note: Authors migrating a paper from traditional SIGPLAN
%% proceedings format to PACMPL format must update the
%% '\documentclass' and topmatter commands above; see
%% 'acmart-pacmpl-template.tex'.
%%%%%%%%%%%%%%%%%%%%%%%%%%%%%%%%%%%%%%%%%%%%%%%%%%%%%%%%%%%%%%%%%%%%%%


%% Some recommended packages.
\usepackage{booktabs}   %% For formal tables:
                        %% http://ctan.org/pkg/booktabs
\usepackage{subcaption} %% For complex figures with subfigures/subcaptions
                        %% http://ctan.org/pkg/subcaption


\begin{document}

%% Title information
\title{Phonograph: A Record-and-Replay Framework for Debugging Serverless
Applications}
% \title[Short Title]{Full Title}         %% [Short Title] is optional;
%                                         %% when present, will be used in
%                                         %% header instead of Full Title.
% \titlenote{with title note}             %% \titlenote is optional;
%                                         %% can be repeated if necessary;
%                                         %% contents suppressed with 'anonymous'
% \subtitle{Subtitle}                     %% \subtitle is optional
% \subtitlenote{with subtitle note}       %% \subtitlenote is optional;
%                                         %% can be repeated if necessary;
%                                         %% contents suppressed with 'anonymous'


%% Author information
%% Contents and number of authors suppressed with 'anonymous'.
%% Each author should be introduced by \author, followed by
%% \authornote (optional), \orcid (optional), \affiliation, and
%% \email.
%% An author may have multiple affiliations and/or emails; repeat the
%% appropriate command.
%% Many elements are not rendered, but should be provided for metadata
%% extraction tools.

%% Author with single affiliation.
\author{First1 Last1}
\authornote{with author1 note}          %% \authornote is optional;
                                        %% can be repeated if necessary
\orcid{nnnn-nnnn-nnnn-nnnn}             %% \orcid is optional
\affiliation{
  \position{Position1}
  \department{Department1}              %% \department is recommended
  \institution{Institution1}            %% \institution is required
  \streetaddress{Street1 Address1}
  \city{City1}
  \state{State1}
  \postcode{Post-Code1}
  \country{Country1}                    %% \country is recommended
}
\email{first1.last1@inst1.edu}          %% \email is recommended

%% Author with two affiliations and emails.
\author{First2 Last2}
\authornote{with author2 note}          %% \authornote is optional;
                                        %% can be repeated if necessary
\orcid{nnnn-nnnn-nnnn-nnnn}             %% \orcid is optional
\affiliation{
  \position{Position2a}
  \department{Department2a}             %% \department is recommended
  \institution{Institution2a}           %% \institution is required
  \streetaddress{Street2a Address2a}
  \city{City2a}
  \state{State2a}
  \postcode{Post-Code2a}
  \country{Country2a}                   %% \country is recommended
}
\email{first2.last2@inst2a.com}         %% \email is recommended
\affiliation{
  \position{Position2b}
  \department{Department2b}             %% \department is recommended
  \institution{Institution2b}           %% \institution is required
  \streetaddress{Street3b Address2b}
  \city{City2b}
  \state{State2b}
  \postcode{Post-Code2b}
  \country{Country2b}                   %% \country is recommended
}
\email{first2.last2@inst2b.org}         %% \email is recommended


\begin{abstract}

The rapid adoption of serverless computing by industry has been marred by a
lack of adequate tooling for software development, with the most notable
deficiencies in tools for monitoring and debugging serverless application
deployed in public-cloud production environments.

While several attempts to provide debug and monitoring solutions for serverless
applications had been made by cloud providers and startups, a viable solution
has yet to emerge. Consequently, most users resort to such bronze-age
techniques as logging and finger-crossing to analyze executions.

The lack of proper debug and monitoring tools is further compounded by the
prevalent use of cloud-based distributed services, such as distributed
databases that provide relaxed consistency guarantees, which are hard enough to
debug in traditional environments.

In this work we present a debugging technique for serverless applications,
which combines run-time recording of function execution, with postmortem replay
capabilities and a novel technique for root-cause analysis. Our system
automatically detects potentially inconsistent reads in distributed databases,
and gives users the ability to analyze the state of the database at various
times in the execution of the application, as well as tie the each value the
database to the specific function execution that produced it.

\end{abstract}



%% 2012 ACM Computing Classification System (CSS) concepts
%% Generate at 'http://dl.acm.org/ccs/ccs.cfm'.
\begin{CCSXML}
<ccs2012>
<concept>
<concept_id>10011007.10011006.10011008</concept_id>
<concept_desc>Software and its engineering~General programming languages</concept_desc>
<concept_significance>500</concept_significance>
</concept>
<concept>
<concept_id>10003456.10003457.10003521.10003525</concept_id>
<concept_desc>Social and professional topics~History of programming languages</concept_desc>
<concept_significance>300</concept_significance>
</concept>
</ccs2012>
\end{CCSXML}

\ccsdesc[500]{Software and its engineering~General programming languages}
\ccsdesc[300]{Social and professional topics~History of programming languages}
%% End of generated code


%% Keywords
%% comma separated list
% \keywords{keyword1, keyword2, keyword3}  %% \keywords are mandatory in final camera-ready submission


%% \maketitle
%% Note: \maketitle command must come after title commands, author
%% commands, abstract environment, Computing Classification System
%% environment and commands, and keywords command.
\maketitle

\section{Introduction}

Recent years have a seen a rapid growth in the adoption of serverless computing
by industry. Serverless platforms, pioneered by AWS Lambda, offer a combination
of zero up-front cost, low maintenance, and a pay-as-you-go pricing model that
allows low-cost software deployment % Cite papers on serverless savings.

Serverless platforms, also known as Functions-as-a-Service (FaaS), are cloud
computing services that let users upload functions to the cloud, and define
events, such as http requests, that trigger those functions. The cloud provider
is then responsible for monitoring events, instantiating the execution
environment (e.g. spinning up a vm that matches the function requirements), and
invoking the function on the event inputs. 

In order to facilitate this execution model, serverless functions must conform
to some technical restrictions: serverless functions must be stateless, and
short-lived. Serverless platforms do not make any guarantees on the reuse of
execution environments, and consequently, a developer cannot rely on the state
of the execution being preserved across invocations. Additionally, for technical
reasons serverless function execution running times are capped (300s for most
cloud providers).

The main difference between serverless applications and traditional, monolithic
applications is the event-driven programming model. While event-driven
programming exists since at least the early 90s, it has not gained popularity
outside of the domains of GUI and hardware programming. The main barrier for
adoption being the the non-standard programming model, and the complexity of
debugging event-driven applications. The vicious cycle of lack of popularity
that leads to a lack of proper tooling that would otherwise ease the development
process certainly did not help.

The advent of serverless computing, with its myriad of benefits, seems to be
forcing the hand of programmers and finally pushing event-driven programming
into the mainstream. Once software developers migrate to serverless computing,
however, they are forced to contend with this lack of debug tools, on top of the
need to switch programming paradigms. Consequently, they tend to resort to the
most basic debug technique know to man---log printouts.

The already-hard problem of debugging event-driven application is made
significantly harder by the setting of serverless computing. The ephemeral
nature of serverless function executions means that it is impossible to
determine, a-priori, on which machine a function will be executed. This makes
traditional techniques of remotely connecting to a running execution impossible
to apply to the serverless setting without explicit support from the platform
providers. Additionally, strict function timeouts mean that even if you
successfully connect to a running function, debugging it is a race against the
clock. Making matters even worse is the fact that no provider currently gives
the option of freezing the execution of the entire system, as is common when
debugging traditional multi-threaded applications. Thus, it is impossible to get
a snapshot of the whole system state, which significantly complicates the task
of understanding the behaviour of the system as a whole.

The reliance on databases to store the program state of serverless applications
brings to the forefront the problem of data provenance tracking---the ability to
identify, given a datum, the specific function invocation that produced that
datum. Consider for example a debug session in which the programmer steps
through an execution of a serverless function, and encounters a failure that is
the result of an unexpected value stored in the database. In order to continue
the debug process and discover the root cause of this failure, the user needs to
determine what was the function execution that wrote this unexpected value to
the database, and debug that function. In a local setting, where most shared
state is stored in memory, all a user would have to do is rerun the execution
with a conditional breakpoint that breaks the execution when a specific value is
written to a specific variable. However, the serverless setting does not provide
any way to rerun the execution (even when the bug is reproducible), or add
variable watches or breakpoints. Following such a sequence of data provenance
investigations to the root cause of the bug is a near-sisyphean undertaking.

Finally, the cloud setting usually involves the use of cloud services, which
often provide relaxed consistency guarantees. For example, amazon's DynamoDB
provides an eventual-consistency wherein data writes \emph{usually} propagate
within 1 second. These consistency guarantees may lead to subtle bugs that occur
or as a result of unexpected function invocation timings. This is true even when
developers are fully aware of the consistency guarantees and plans for them, let
alone in the arguably more common case of developers failing to account for
unconventional consistency guarantees. 

In this work we set about to closing this gap. We introduce \system, a
\emph{record and replay} system for recording the execution of serverless
applications in active deployment. \system lets the user rerun locally any
function invocation in a way that faithfully recreates the function execution on
the cloud. The user can then use a debugger to debug the execution of the
function in the same way they would debug a monolithic application. When the
user encounters an unexpected value in the database (an unexpected result of a
database read call) \system provides the capabilities to jump to the function
that originally wrote that to the database, and debug that function.
Additionally, \system automatically detects inconsistent reads that may have
resulted from the lax consistency guarantees made by the system, thus bringing
this source of potential problems to the forefront of developer's attention.



% %% Acknowledgments
\begin{acks}                            %% acks environment is optional
                                        %% contents suppressed with 'anonymous'
  %% Commands \grantsponsor{<sponsorID>}{<name>}{<url>} and
  %% \grantnum[<url>]{<sponsorID>}{<number>} should be used to
  %% acknowledge financial support and will be used by metadata
  %% extraction tools.
  This material is based upon work supported by the
  \grantsponsor{GS100000001}{National Science
    Foundation}{http://dx.doi.org/10.13039/100000001} under Grant
  No.~\grantnum{GS100000001}{nnnnnnn} and Grant
  No.~\grantnum{GS100000001}{mmmmmmm}.  Any opinions, findings, and
  conclusions or recommendations expressed in this material are those
  of the author and do not necessarily reflect the views of the
  National Science Foundation.
\end{acks}



%% Bibliography
\bibliography{refs}

% %% Appendix
\appendix
\section{Appendix}

Text of appendix \ldots


\end{document}
